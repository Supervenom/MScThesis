%********************************************************************
% Appendix
%*******************************************************
% If problems with the headers: get headings in appendix etc. right
%\markboth{\spacedlowsmallcaps{Appendix}}{\spacedlowsmallcaps{Appendix}}
\chapter{Induction Proof Principles}
In this appendix some aspects of induction proof principles given for granted in the remainder of this thesis will be clarified, with precise definitions and formal statements.
\begin{definition}\hfill
	\begin{enumerate}
	\item A relation $\prec$ is \emph{well-founded} on a set $X$ if for every non-empty subset $S$ of $X$, $S$ has an  element $x$ such that there exists no $y$ in $X$ satisfying $y\prec x$.
	\item A relation $\prec$ is well-founded on a set $X$ if it contains no infinite descending chains i.e. infinite sequences $(x_i)_{i\in\mathbb{N}}$ of elements of $X$ such that $x_{i+1}\prec x_i$ for each $i$ in $\mathbb{N}$.
	\end{enumerate}
\end{definition}
Assuming the \emph{axiom of choice} one can prove that the two definitions above are equivalent.
\begin{proof}[Proof of Equivalence]
	(1 $\Rightarrow$ 2). If there would be an infinite sequence $(x_i)_{i\in\mathbb{N}}$ of elements of $X$ such that $x_{i+1}\prec x_i$ for each $i\in\mathbb{N}$, then the set $Y=\{x_i\,|\,i\in\mathbb{N}\}$ would not satisfy $1$.\\
	(2 $\Rightarrow$ 1). If $\prec$ does not satisfy 1, then there is a non-empty subset $S$ of $X$ such that for every $x\in S$ there exists $y\in X$ such that $y\prec x$. Choose $x_0$ in $S$. Then choose $x_1$ such that $x_1\prec x_0$. Proceeding this way we can form an infinite sequence $(x_i)_{i\in\mathbb{N}}$ of elements of $X$ such that $x_{i+1}\prec x_i$ for each $i$ in $\mathbb{N}$.
\end{proof}
Well-founded relations are interesting because they allow proofs by induction.
\begin{proposition}
	Let $(X,\prec)$ be a set endowed with a well-founded relation and $\mathbf{P}$ a property on the elements of $X$. If for each $x\in X$ it holds that: 
	\begin{center} if $\mathbf{P}(y)$ holds  for each $y\prec x$ then $\mathbf{P}(x)$\end{center}
	then $\mathbf{P}(x)$ holds for every $x\in X$.
\end{proposition}
\begin{proof}
\end{proof}

