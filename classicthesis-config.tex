% ****************************************************************************************************
% classicthesis-config.tex
% formerly known as loadpackages.sty, classicthesis-ldpkg.sty, and classicthesis-preamble.sty
% Use it at the beginning of your ClassicThesis.tex, or as a LaTeX Preamble
% in your ClassicThesis.{tex,lyx} with % ****************************************************************************************************
% classicthesis-config.tex
% formerly known as loadpackages.sty, classicthesis-ldpkg.sty, and classicthesis-preamble.sty
% Use it at the beginning of your ClassicThesis.tex, or as a LaTeX Preamble
% in your ClassicThesis.{tex,lyx} with % ****************************************************************************************************
% classicthesis-config.tex
% formerly known as loadpackages.sty, classicthesis-ldpkg.sty, and classicthesis-preamble.sty
% Use it at the beginning of your ClassicThesis.tex, or as a LaTeX Preamble
% in your ClassicThesis.{tex,lyx} with % ****************************************************************************************************
% classicthesis-config.tex
% formerly known as loadpackages.sty, classicthesis-ldpkg.sty, and classicthesis-preamble.sty
% Use it at the beginning of your ClassicThesis.tex, or as a LaTeX Preamble
% in your ClassicThesis.{tex,lyx} with \input{classicthesis-config}
% ****************************************************************************************************
% If you like the classicthesis, then I would appreciate a postcard.
% My address can be found in the file ClassicThesis.pdf. A collection
% of the postcards I received so far is available online at
% http://postcards.miede.de
% ****************************************************************************************************


% ****************************************************************************************************
% 0. Set the encoding of your files. UTF-8 is the only sensible encoding nowadays. If you can't read
% äöüßáéçèê∂åëæƒÏ€ then change the encoding setting in your editor, not the line below. If your editor
% does not support utf8 use another editor!
% ****************************************************************************************************
\PassOptionsToPackage{utf8}{inputenc}
  \usepackage{inputenc}

% ****************************************************************************************************
% 1. Configure classicthesis for your needs here, e.g., remove "drafting" below
% in order to deactivate the time-stamp on the pages
% (see ClassicThesis.pdf for more information):
% ****************************************************************************************************
\PassOptionsToPackage{
  drafting=false,    % print version information on the bottom of the pages
  tocaligned=false, % the left column of the toc will be aligned (no indentation)
  dottedtoc=false,  % page numbers in ToC flushed right
  parts=false,       % use part division
  eulerchapternumbers=true, % use AMS Euler for chapter font (otherwise Palatino)
  linedheaders=true,       % chaper headers will have line above and beneath
  floatperchapter=true,     % numbering per chapter for all floats (i.e., Figure 1.1)
  listings=true,    % load listings package and setup LoL
  subfig=true,      % setup for preloaded subfig package
  eulermath=true,  % use awesome Euler fonts for mathematical formulae (only with pdfLaTeX)
  beramono=true,    % toggle a nice monospaced font (w/ bold)
  minionpro=false   % setup for minion pro font; use minion pro small caps as well (only with pdfLaTeX)
}{classicthesis}


% ****************************************************************************************************
% 2. Personal data and user ad-hoc commands
% ****************************************************************************************************
\newcommand{\myTitle}{On Randomised Strategies in the $\lambda$-calculus\xspace}
\newcommand{\mySubtitle}{}
\newcommand{\myDegree}{Dottor Ingegnere (Dott. Ing.)\xspace}
\newcommand{\myName}{Gabriele Vanoni\xspace}
\newcommand{\myProf}{Matteo Pradella\xspace}
\newcommand{\myOtherProf}{Ugo Dal Lago\xspace}
\newcommand{\mySupervisor}{Put name here\xspace}
\newcommand{\myFaculty}{Scuola di Ingegneria Industriale e dell'Informazione\xspace}
\newcommand{\myDepartment}{Corso di Laurea Magistrale in Ingegneria Informatica\xspace}
\newcommand{\myUni}{Politecnico di Milano\xspace}
\newcommand{\myLocation}{Milano\xspace}
\newcommand{\myTime}{July 2018\xspace}
\newcommand{\myVersion}{version 4.4}

% ********************************************************************
% Setup, finetuning, and useful commands
% ********************************************************************
\newcounter{dummy} % necessary for correct hyperlinks (to index, bib, etc.)
\newlength{\abcd} % for ab..z string length calculation
\providecommand{\mLyX}{L\kern-.1667em\lower.25em\hbox{Y}\kern-.125emX\@}
\newcommand{\ie}{i.\,e.}
\newcommand{\Ie}{I.\,e.}
\newcommand{\eg}{e.\,g.}
\newcommand{\Eg}{E.\,g.}
% ****************************************************************************************************


% ****************************************************************************************************
% 3. Loading some handy packages
% ****************************************************************************************************
% ********************************************************************
% Packages with options that might require adjustments
% ********************************************************************
%\PassOptionsToPackage{ngerman,american}{babel}   % change this to your language(s), main language last
% Spanish languages need extra options in order to work with this template
%\PassOptionsToPackage{spanish,es-lcroman}{babel}
    \usepackage{babel}
\usepackage{csquotes}
\PassOptionsToPackage{%
  %backend=biber,bibencoding=utf8, %instead of bibtex
  backend=bibtex8,bibencoding=ascii,%
  language=auto,%
  style=numeric-comp,%
  %style=authoryear-comp, % Author 1999, 2010
  %citestyle=authoryear,
  %bibstyle=authoryear,dashed=false, % dashed: substitute rep. author with ---
  sorting=nyt, % name, year, title
  maxbibnames=10, % default: 3, et al.
  %backref=true,%
  natbib=true % natbib compatibility mode (\citep and \citet still work)
}{biblatex}
    \usepackage{biblatex}

\PassOptionsToPackage{fleqn}{amsmath}       % math environments and more by the AMS
  \usepackage{amsmath}

% ********************************************************************
% General useful packages
% ********************************************************************
\PassOptionsToPackage{T1}{fontenc} % T2A for cyrillics
  \usepackage{fontenc}
\usepackage{textcomp} % fix warning with missing font shapes
\usepackage{scrhack} % fix warnings when using KOMA with listings package
\usepackage{xspace} % to get the spacing after macros right
\usepackage{mparhack} % get marginpar right
%\usepackage{fixltx2e} % fixes some LaTeX stuff --> since 2015 in the LaTeX kernel (see below)
% \usepackage[latest]{latexrelease} % emulate newer kernel version if older is detected
\PassOptionsToPackage{printonlyused,smaller}{acronym}
  \usepackage{acronym} % nice macros for handling all acronyms in the thesis
  %\renewcommand{\bflabel}[1]{{#1}\hfill} % fix the list of acronyms --> no longer working
  %\renewcommand*{\acsfont}[1]{\textsc{#1}}
  %\renewcommand*{\aclabelfont}[1]{\acsfont{#1}}
  %\def\bflabel#1{{#1\hfill}}
  \def\bflabel#1{{\acsfont{#1}\hfill}}
  \def\aclabelfont#1{\acsfont{#1}}
% ****************************************************************************************************
%\usepackage{pgfplots} % External TikZ/PGF support (thanks to Andreas Nautsch)
%\usetikzlibrary{external}
%\tikzexternalize[mode=list and make, prefix=ext-tikz/]
% ****************************************************************************************************


% ****************************************************************************************************
% 4. Setup floats: tables, (sub)figures, and captions
% ****************************************************************************************************
\usepackage{tabularx} % better tables
  \setlength{\extrarowheight}{3pt} % increase table row height
\newcommand{\tableheadline}[1]{\multicolumn{1}{c}{\spacedlowsmallcaps{#1}}}
\newcommand{\myfloatalign}{\centering} % to be used with each float for alignment
\usepackage{caption}
% Thanks to cgnieder and Claus Lahiri
% http://tex.stackexchange.com/questions/69349/spacedlowsmallcaps-in-caption-label
% [REMOVED DUE TO OTHER PROBLEMS, SEE ISSUE #82]
%\DeclareCaptionLabelFormat{smallcaps}{\bothIfFirst{#1}{~}\MakeTextLowercase{\textsc{#2}}}
%\captionsetup{font=small,labelformat=smallcaps} % format=hang,
\captionsetup{font=small} % format=hang,
\usepackage{subfig}
% ****************************************************************************************************


% ****************************************************************************************************
% 5. Setup code listings
% ****************************************************************************************************
\usepackage{listings}
%\lstset{emph={trueIndex,root},emphstyle=\color{BlueViolet}}%\underbar} % for special keywords
\lstset{language=[LaTeX]Tex,%C++,
  morekeywords={PassOptionsToPackage,selectlanguage},
  keywordstyle=\color{RoyalBlue},%\bfseries,
  basicstyle=\small\ttfamily,
  %identifierstyle=\color{NavyBlue},
  commentstyle=\color{Green}\ttfamily,
  stringstyle=\rmfamily,
  numbers=none,%left,%
  numberstyle=\scriptsize,%\tiny
  stepnumber=5,
  numbersep=8pt,
  showstringspaces=false,
  breaklines=true,
  %frameround=ftff,
  %frame=single,
  belowcaptionskip=.75\baselineskip
  %frame=L
}
% ****************************************************************************************************


% ****************************************************************************************************
% 6. PDFLaTeX, hyperreferences, and citation backreferences
% ****************************************************************************************************
% ********************************************************************
% Using PDFLaTeX
% ********************************************************************
\PassOptionsToPackage{hyperfootnotes=false,pdfpagelabels}{hyperref}
  \usepackage{hyperref}  % backref linktocpage pagebackref
%\ifpdf
%\pdfcompresslevel=9
%\pdfadjustspacing=1
%\fi
%\PassOptionsToPackage{pdftex}{graphicx} %%%IVO: driver will be chosen automatically
  \usepackage{graphicx}


% ********************************************************************
% Hyperreferences
% ********************************************************************
\hypersetup{%
  %draft, % hyperref's draft mode, for printing see below
  colorlinks=true, linktocpage=true, pdfstartpage=3, pdfstartview=FitV,%
  % uncomment the following line if you want to have black links (e.g., for printing)
  %colorlinks=false, linktocpage=false, pdfstartpage=3, pdfstartview=FitV, pdfborder={0 0 0},%
  breaklinks=true, pdfpagemode=UseNone, pageanchor=true, pdfpagemode=UseOutlines,%
  plainpages=false, bookmarksnumbered, bookmarksopen=true, bookmarksopenlevel=1,%
  hypertexnames=true, pdfhighlight=/O,%nesting=true,%frenchlinks,%
  urlcolor=webbrown, linkcolor=RoyalBlue, citecolor=webgreen, %pagecolor=RoyalBlue,%
  %urlcolor=Black, linkcolor=Black, citecolor=Black, %pagecolor=Black,%
  pdftitle={\myTitle},%
  pdfauthor={\textcopyright\ \myName, \myUni, \myFaculty},%
  pdfsubject={},%
  pdfkeywords={},%
  pdfcreator={pdfLaTeX},%
  pdfproducer={LaTeX with hyperref and classicthesis}%
}

% ********************************************************************
% Setup autoreferences
% ********************************************************************
% There are some issues regarding autorefnames
% http://www.ureader.de/msg/136221647.aspx
% http://www.tex.ac.uk/cgi-bin/texfaq2html?label=latexwords
% you have to redefine the makros for the
% language you use, e.g., american, ngerman
% (as chosen when loading babel/AtBeginDocument)
% ********************************************************************
\makeatletter
\@ifpackageloaded{babel}%
  {%
    \addto\extrasamerican{%
      \renewcommand*{\figureautorefname}{Figure}%
      \renewcommand*{\tableautorefname}{Table}%
      \renewcommand*{\partautorefname}{Part}%
      \renewcommand*{\chapterautorefname}{Chapter}%
      \renewcommand*{\sectionautorefname}{Section}%
      \renewcommand*{\subsectionautorefname}{Section}%
      \renewcommand*{\subsubsectionautorefname}{Section}%
    }%
    \addto\extrasngerman{%
      \renewcommand*{\paragraphautorefname}{Absatz}%
      \renewcommand*{\subparagraphautorefname}{Unterabsatz}%
      \renewcommand*{\footnoteautorefname}{Fu\"snote}%
      \renewcommand*{\FancyVerbLineautorefname}{Zeile}%
      \renewcommand*{\theoremautorefname}{Theorem}%
      \renewcommand*{\appendixautorefname}{Anhang}%
      \renewcommand*{\equationautorefname}{Gleichung}%
      \renewcommand*{\itemautorefname}{Punkt}%
    }%
      % Fix to getting autorefs for subfigures right (thanks to Belinda Vogt for changing the definition)
      \providecommand{\subfigureautorefname}{\figureautorefname}%
    }{\relax}
\makeatother


% ****************************************************************************************************
% 7. Last calls before the bar closes
% ****************************************************************************************************
% ********************************************************************
% Development Stuff
% ********************************************************************
\listfiles
%\PassOptionsToPackage{l2tabu,orthodox,abort}{nag}
%  \usepackage{nag}
%\PassOptionsToPackage{warning, all}{onlyamsmath}
%  \usepackage{onlyamsmath}

% ********************************************************************
% Last, but not least...
% ********************************************************************
\usepackage{classicthesis}
% ****************************************************************************************************


% ****************************************************************************************************
% 8. Further adjustments (experimental)
% ****************************************************************************************************
% ********************************************************************
% Changing the text area
% ********************************************************************
%\areaset[current]{312pt}{761pt} % 686 (factor 2.2) + 33 head + 42 head \the\footskip
%\setlength{\marginparwidth}{7em}%
%\setlength{\marginparsep}{2em}%

% ********************************************************************
% Using different fonts
% ********************************************************************
%\usepackage[oldstylenums]{kpfonts} % oldstyle notextcomp
%\usepackage[osf]{libertine}
%\usepackage[light,condensed,math]{iwona}
%\renewcommand{\sfdefault}{iwona}
%\usepackage{lmodern} % <-- no osf support :-(
%\usepackage{cfr-lm} %
%\usepackage[urw-garamond]{mathdesign} <-- no osf support :-(
%\usepackage[default,osfigures]{opensans} % scale=0.95
%\usepackage[sfdefault]{FiraSans}
% ********************************************************************
% \usepackage[largesc,osf]{newpxtext}
% Used to fix these:
% https://bitbucket.org/amiede/classicthesis/issues/139/italics-in-pallatino-capitals-chapter
% https://bitbucket.org/amiede/classicthesis/issues/45/problema-testatine-su-classicthesis-style
% ********************************************************************
%\linespread{1.05} % a bit more for Palatino
% ****************************************************************************************************

%%% SYNTAX %%%
\newcommand{\bnf}{::=}
\newcommand{\midd}{\; \; \mbox{\Large{$\mid$}}\;\;}

%%% TERMS %%%
\newcommand{\termone}{M}
\newcommand{\termtwo}{N}
\newcommand{\termthree}{L}
\newcommand{\termfour}{P}
\newcommand{\termfive}{S}
\newcommand{\termsix}{V}
\newcommand{\termseven}{A}
\newcommand{\termeight}{B}
\newcommand{\varone}{x}
\newcommand{\abstr}[2]{\lambda #1.#2}
\newcommand{\subst}[3]{#1\{#2/#3\}}
\newcommand{\rdxone}{R}
\newcommand{\rdxtwo}{Q}
\newcommand{\rdxs}[1]{\mathcal{R}_{#1}}
\newcommand{\rdxset}{\mathcal{R}}
\newcommand{\crdxs}[1]{\mathcal{N}_{#1}}

%%% CONTEXTS %%%
\newcommand{\contone}{C}
\newcommand{\conttwo}{D}
\newcommand{\hole}{[\cdot]}

%%% REDUCTION RELATIONS %%%
\newcommand{\redbeta}{\longrightarrow_\beta}
\newcommand{\redbetaclo}{\xtwoheadrightarrow{}_\beta}
\newcommand{\red}{\rightarrow}
\newcommand{\redlo}{\longrightarrow_\pslo}
\newcommand{\redri}{\longrightarrow_\psri}
\newcommand{\redbetardx}[1]{\overset{#1}{\longrightarrow_\beta}}
\newcommand{\redbetared}[1]{\overset{#1}{\xtwoheadrightarrow{}_\beta}}
\newcommand{\redlosteps}[1]{\longrightarrow_\pslo^#1}
\newcommand{\redlordx}[1]{\overset{#1}{\longrightarrow}_\pslo}
\newcommand{\redanf}{\underset{\mathsf{ANF}}{\longrightarrow}}
\newcommand{\redbetaanf}{\underset{\beta\mathsf{ANF}}{\longrightarrow}}
\newcommand{\redbetaanfsteps}[1]{\underset{\beta\mathsf{ANF}}{\longrightarrow^{#1}}}
\newcommand{\redbetasteps}[1]{\longrightarrow^{#1}_\beta}
\newcommand{\redbetaanfrdx}[1]{\overset{#1}{\underset{\beta\mathsf{ANF}}{\longrightarrow}}}

%%% STRATEGIES %%%
\newcommand{\psone}{\mathsf{P}}
\newcommand{\mcp}[1]{\mathcal{M}_{#1}}
\newcommand{\density}[1]{{f}_{#1}}
\newcommand{\densityp}[2]{{f}_{#1}^{#2}}
\newcommand{\psuni}{\mathsf{U}}
\newcommand{\pslo}{\mathsf{LO}}
\newcommand{\psri}{\mathsf{RI}}
\newcommand{\psparam}[1]{\mathsf{P}(#1)}
\newcommand{\psparamtwo}[1]{\mathsf{P_2}(#1)}
\newcommand{\psgeo}{\mathsf{G}}
\newcommand{\dist}[1]{\mathsf{Dist}\left(#1\right)}
\newcommand{\supp}[1]{\mathbf{Supp}\left(#1\right)}
\newcommand{\pdist}[1]{\mathsf{PDist}\left(#1\right)}
\newcommand{\derlenght}[1]{\mathsf{dl_\mathcal{#1}}}
\newcommand{\avglenght}[1]{\mathsf{adl_\mathcal{#1}}}

%%% MISC %%%
\newcommand{\depth}[2]{d_{#1}(#2)}
\newcommand{\nsteps}[1]{\mathsf{Steps}_{#1}}
\newcommand{\pseries}[1]{\mathsf{PS}(#1)}
\newcommand{\poly}[1]{\mathsf{Poly}(#1)}

\newenvironment{varitemize}
{
	\begin{list}{\labelitemi}
		{\setlength{\itemsep}{0pt}
			\setlength{\topsep}{0pt}
			\setlength{\parsep}{0pt}
			\setlength{\partopsep}{0pt}
			\setlength{\leftmargin}{15pt}
			\setlength{\rightmargin}{0pt}
			\setlength{\itemindent}{0pt}
			\setlength{\labelsep}{5pt}
			\setlength{\labelwidth}{10pt}
		}}
		{
	\end{list}
}

\usepackage{extpfeil}
\usepackage{bm}
\usepackage{ifthen}
\usepackage{amssymb,amsthm}
\usepackage{microtype}
\usepackage{tikz}
\usetikzlibrary{trees}
\usepackage{rotating}
\usetikzlibrary{automata,positioning}
\renewcommand{\labelitemi}{$\bullet$}
%%%  TREES %%%
\tikzstyle{level 1}=[level distance=2.5cm, sibling distance=4.5cm]
\tikzstyle{level 2}=[level distance=2.3cm, sibling distance=4.5cm]
\tikzstyle{term} = [text centered]
\newboolean{lv}
\setboolean{lv}{true}
\newboolean{verbose}
\setboolean{verbose}{true}


%%% MATH %%%
\theoremstyle{plain}
\newtheorem{theorem}{\protect\theoremname}
\theoremstyle{definition}
\newtheorem{definition}[theorem]{\protect\definitionname}
\theoremstyle{definition}
%\newtheorem{example}[theorem]{\protect\examplename}
\theoremstyle{definition}
\newtheorem{lemma}[theorem]{\protect\lemmaname}
\theoremstyle{definition}
\newtheorem{corollary}[theorem]{\protect\corollaryname}
\theoremstyle{definition}
\newtheorem{proposition}[theorem]{\protect\propositionname}
\theoremstyle{definition}
%\newtheorem*{example*}{\protect\examplename}
\newtheorem*{proposition*}{\protect\propositionname}
\newtheorem*{corollary*}{\protect\corollaryname}
\newtheorem*{lemma*}{\protect\lemmaname}
\newtheorem*{theorem*}{\protect\theoremname}
\newtheorem*{remark*}{\protect\remarkname}
\newtheorem{strategy}[theorem]{\protect\strategyname}
%\providecommand{\examplename}{Example}
\providecommand{\strategyname}{Strategy}
\providecommand{\theoremname}{Theorem}
\providecommand{\propositionname}{Proposition}
\providecommand{\corollaryname}{Corollary}
\providecommand{\lemmaname}{Lemma}
\providecommand{\theoremname}{Theorem}
\providecommand{\definitionname}{Definition}
\providecommand{\remarkname}{Remark}
\newtheorem{examplex}[theorem]{Example}
\newenvironment{example}
{\pushQED{\qed}\renewcommand{\qedsymbol}{$\triangle$}\examplex}
{\popQED\endexamplex}
% ****************************************************************************************************
% If you like the classicthesis, then I would appreciate a postcard.
% My address can be found in the file ClassicThesis.pdf. A collection
% of the postcards I received so far is available online at
% http://postcards.miede.de
% ****************************************************************************************************


% ****************************************************************************************************
% 0. Set the encoding of your files. UTF-8 is the only sensible encoding nowadays. If you can't read
% äöüßáéçèê∂åëæƒÏ€ then change the encoding setting in your editor, not the line below. If your editor
% does not support utf8 use another editor!
% ****************************************************************************************************
\PassOptionsToPackage{utf8}{inputenc}
  \usepackage{inputenc}

% ****************************************************************************************************
% 1. Configure classicthesis for your needs here, e.g., remove "drafting" below
% in order to deactivate the time-stamp on the pages
% (see ClassicThesis.pdf for more information):
% ****************************************************************************************************
\PassOptionsToPackage{
  drafting=false,    % print version information on the bottom of the pages
  tocaligned=false, % the left column of the toc will be aligned (no indentation)
  dottedtoc=false,  % page numbers in ToC flushed right
  parts=false,       % use part division
  eulerchapternumbers=true, % use AMS Euler for chapter font (otherwise Palatino)
  linedheaders=true,       % chaper headers will have line above and beneath
  floatperchapter=true,     % numbering per chapter for all floats (i.e., Figure 1.1)
  listings=true,    % load listings package and setup LoL
  subfig=true,      % setup for preloaded subfig package
  eulermath=true,  % use awesome Euler fonts for mathematical formulae (only with pdfLaTeX)
  beramono=true,    % toggle a nice monospaced font (w/ bold)
  minionpro=false   % setup for minion pro font; use minion pro small caps as well (only with pdfLaTeX)
}{classicthesis}


% ****************************************************************************************************
% 2. Personal data and user ad-hoc commands
% ****************************************************************************************************
\newcommand{\myTitle}{On Randomised Strategies in the $\lambda$-calculus\xspace}
\newcommand{\mySubtitle}{}
\newcommand{\myDegree}{Dottor Ingegnere (Dott. Ing.)\xspace}
\newcommand{\myName}{Gabriele Vanoni\xspace}
\newcommand{\myProf}{Matteo Pradella\xspace}
\newcommand{\myOtherProf}{Ugo Dal Lago\xspace}
\newcommand{\mySupervisor}{Put name here\xspace}
\newcommand{\myFaculty}{Scuola di Ingegneria Industriale e dell'Informazione\xspace}
\newcommand{\myDepartment}{Corso di Laurea Magistrale in Ingegneria Informatica\xspace}
\newcommand{\myUni}{Politecnico di Milano\xspace}
\newcommand{\myLocation}{Milano\xspace}
\newcommand{\myTime}{July 2018\xspace}
\newcommand{\myVersion}{version 4.4}

% ********************************************************************
% Setup, finetuning, and useful commands
% ********************************************************************
\newcounter{dummy} % necessary for correct hyperlinks (to index, bib, etc.)
\newlength{\abcd} % for ab..z string length calculation
\providecommand{\mLyX}{L\kern-.1667em\lower.25em\hbox{Y}\kern-.125emX\@}
\newcommand{\ie}{i.\,e.}
\newcommand{\Ie}{I.\,e.}
\newcommand{\eg}{e.\,g.}
\newcommand{\Eg}{E.\,g.}
% ****************************************************************************************************


% ****************************************************************************************************
% 3. Loading some handy packages
% ****************************************************************************************************
% ********************************************************************
% Packages with options that might require adjustments
% ********************************************************************
%\PassOptionsToPackage{ngerman,american}{babel}   % change this to your language(s), main language last
% Spanish languages need extra options in order to work with this template
%\PassOptionsToPackage{spanish,es-lcroman}{babel}
    \usepackage{babel}
\usepackage{csquotes}
\PassOptionsToPackage{%
  %backend=biber,bibencoding=utf8, %instead of bibtex
  backend=bibtex8,bibencoding=ascii,%
  language=auto,%
  style=numeric-comp,%
  %style=authoryear-comp, % Author 1999, 2010
  %citestyle=authoryear,
  %bibstyle=authoryear,dashed=false, % dashed: substitute rep. author with ---
  sorting=nyt, % name, year, title
  maxbibnames=10, % default: 3, et al.
  %backref=true,%
  natbib=true % natbib compatibility mode (\citep and \citet still work)
}{biblatex}
    \usepackage{biblatex}

\PassOptionsToPackage{fleqn}{amsmath}       % math environments and more by the AMS
  \usepackage{amsmath}

% ********************************************************************
% General useful packages
% ********************************************************************
\PassOptionsToPackage{T1}{fontenc} % T2A for cyrillics
  \usepackage{fontenc}
\usepackage{textcomp} % fix warning with missing font shapes
\usepackage{scrhack} % fix warnings when using KOMA with listings package
\usepackage{xspace} % to get the spacing after macros right
\usepackage{mparhack} % get marginpar right
%\usepackage{fixltx2e} % fixes some LaTeX stuff --> since 2015 in the LaTeX kernel (see below)
% \usepackage[latest]{latexrelease} % emulate newer kernel version if older is detected
\PassOptionsToPackage{printonlyused,smaller}{acronym}
  \usepackage{acronym} % nice macros for handling all acronyms in the thesis
  %\renewcommand{\bflabel}[1]{{#1}\hfill} % fix the list of acronyms --> no longer working
  %\renewcommand*{\acsfont}[1]{\textsc{#1}}
  %\renewcommand*{\aclabelfont}[1]{\acsfont{#1}}
  %\def\bflabel#1{{#1\hfill}}
  \def\bflabel#1{{\acsfont{#1}\hfill}}
  \def\aclabelfont#1{\acsfont{#1}}
% ****************************************************************************************************
%\usepackage{pgfplots} % External TikZ/PGF support (thanks to Andreas Nautsch)
%\usetikzlibrary{external}
%\tikzexternalize[mode=list and make, prefix=ext-tikz/]
% ****************************************************************************************************


% ****************************************************************************************************
% 4. Setup floats: tables, (sub)figures, and captions
% ****************************************************************************************************
\usepackage{tabularx} % better tables
  \setlength{\extrarowheight}{3pt} % increase table row height
\newcommand{\tableheadline}[1]{\multicolumn{1}{c}{\spacedlowsmallcaps{#1}}}
\newcommand{\myfloatalign}{\centering} % to be used with each float for alignment
\usepackage{caption}
% Thanks to cgnieder and Claus Lahiri
% http://tex.stackexchange.com/questions/69349/spacedlowsmallcaps-in-caption-label
% [REMOVED DUE TO OTHER PROBLEMS, SEE ISSUE #82]
%\DeclareCaptionLabelFormat{smallcaps}{\bothIfFirst{#1}{~}\MakeTextLowercase{\textsc{#2}}}
%\captionsetup{font=small,labelformat=smallcaps} % format=hang,
\captionsetup{font=small} % format=hang,
\usepackage{subfig}
% ****************************************************************************************************


% ****************************************************************************************************
% 5. Setup code listings
% ****************************************************************************************************
\usepackage{listings}
%\lstset{emph={trueIndex,root},emphstyle=\color{BlueViolet}}%\underbar} % for special keywords
\lstset{language=[LaTeX]Tex,%C++,
  morekeywords={PassOptionsToPackage,selectlanguage},
  keywordstyle=\color{RoyalBlue},%\bfseries,
  basicstyle=\small\ttfamily,
  %identifierstyle=\color{NavyBlue},
  commentstyle=\color{Green}\ttfamily,
  stringstyle=\rmfamily,
  numbers=none,%left,%
  numberstyle=\scriptsize,%\tiny
  stepnumber=5,
  numbersep=8pt,
  showstringspaces=false,
  breaklines=true,
  %frameround=ftff,
  %frame=single,
  belowcaptionskip=.75\baselineskip
  %frame=L
}
% ****************************************************************************************************


% ****************************************************************************************************
% 6. PDFLaTeX, hyperreferences, and citation backreferences
% ****************************************************************************************************
% ********************************************************************
% Using PDFLaTeX
% ********************************************************************
\PassOptionsToPackage{hyperfootnotes=false,pdfpagelabels}{hyperref}
  \usepackage{hyperref}  % backref linktocpage pagebackref
%\ifpdf
%\pdfcompresslevel=9
%\pdfadjustspacing=1
%\fi
%\PassOptionsToPackage{pdftex}{graphicx} %%%IVO: driver will be chosen automatically
  \usepackage{graphicx}


% ********************************************************************
% Hyperreferences
% ********************************************************************
\hypersetup{%
  %draft, % hyperref's draft mode, for printing see below
  colorlinks=true, linktocpage=true, pdfstartpage=3, pdfstartview=FitV,%
  % uncomment the following line if you want to have black links (e.g., for printing)
  %colorlinks=false, linktocpage=false, pdfstartpage=3, pdfstartview=FitV, pdfborder={0 0 0},%
  breaklinks=true, pdfpagemode=UseNone, pageanchor=true, pdfpagemode=UseOutlines,%
  plainpages=false, bookmarksnumbered, bookmarksopen=true, bookmarksopenlevel=1,%
  hypertexnames=true, pdfhighlight=/O,%nesting=true,%frenchlinks,%
  urlcolor=webbrown, linkcolor=RoyalBlue, citecolor=webgreen, %pagecolor=RoyalBlue,%
  %urlcolor=Black, linkcolor=Black, citecolor=Black, %pagecolor=Black,%
  pdftitle={\myTitle},%
  pdfauthor={\textcopyright\ \myName, \myUni, \myFaculty},%
  pdfsubject={},%
  pdfkeywords={},%
  pdfcreator={pdfLaTeX},%
  pdfproducer={LaTeX with hyperref and classicthesis}%
}

% ********************************************************************
% Setup autoreferences
% ********************************************************************
% There are some issues regarding autorefnames
% http://www.ureader.de/msg/136221647.aspx
% http://www.tex.ac.uk/cgi-bin/texfaq2html?label=latexwords
% you have to redefine the makros for the
% language you use, e.g., american, ngerman
% (as chosen when loading babel/AtBeginDocument)
% ********************************************************************
\makeatletter
\@ifpackageloaded{babel}%
  {%
    \addto\extrasamerican{%
      \renewcommand*{\figureautorefname}{Figure}%
      \renewcommand*{\tableautorefname}{Table}%
      \renewcommand*{\partautorefname}{Part}%
      \renewcommand*{\chapterautorefname}{Chapter}%
      \renewcommand*{\sectionautorefname}{Section}%
      \renewcommand*{\subsectionautorefname}{Section}%
      \renewcommand*{\subsubsectionautorefname}{Section}%
    }%
    \addto\extrasngerman{%
      \renewcommand*{\paragraphautorefname}{Absatz}%
      \renewcommand*{\subparagraphautorefname}{Unterabsatz}%
      \renewcommand*{\footnoteautorefname}{Fu\"snote}%
      \renewcommand*{\FancyVerbLineautorefname}{Zeile}%
      \renewcommand*{\theoremautorefname}{Theorem}%
      \renewcommand*{\appendixautorefname}{Anhang}%
      \renewcommand*{\equationautorefname}{Gleichung}%
      \renewcommand*{\itemautorefname}{Punkt}%
    }%
      % Fix to getting autorefs for subfigures right (thanks to Belinda Vogt for changing the definition)
      \providecommand{\subfigureautorefname}{\figureautorefname}%
    }{\relax}
\makeatother


% ****************************************************************************************************
% 7. Last calls before the bar closes
% ****************************************************************************************************
% ********************************************************************
% Development Stuff
% ********************************************************************
\listfiles
%\PassOptionsToPackage{l2tabu,orthodox,abort}{nag}
%  \usepackage{nag}
%\PassOptionsToPackage{warning, all}{onlyamsmath}
%  \usepackage{onlyamsmath}

% ********************************************************************
% Last, but not least...
% ********************************************************************
\usepackage{classicthesis}
% ****************************************************************************************************


% ****************************************************************************************************
% 8. Further adjustments (experimental)
% ****************************************************************************************************
% ********************************************************************
% Changing the text area
% ********************************************************************
%\areaset[current]{312pt}{761pt} % 686 (factor 2.2) + 33 head + 42 head \the\footskip
%\setlength{\marginparwidth}{7em}%
%\setlength{\marginparsep}{2em}%

% ********************************************************************
% Using different fonts
% ********************************************************************
%\usepackage[oldstylenums]{kpfonts} % oldstyle notextcomp
%\usepackage[osf]{libertine}
%\usepackage[light,condensed,math]{iwona}
%\renewcommand{\sfdefault}{iwona}
%\usepackage{lmodern} % <-- no osf support :-(
%\usepackage{cfr-lm} %
%\usepackage[urw-garamond]{mathdesign} <-- no osf support :-(
%\usepackage[default,osfigures]{opensans} % scale=0.95
%\usepackage[sfdefault]{FiraSans}
% ********************************************************************
% \usepackage[largesc,osf]{newpxtext}
% Used to fix these:
% https://bitbucket.org/amiede/classicthesis/issues/139/italics-in-pallatino-capitals-chapter
% https://bitbucket.org/amiede/classicthesis/issues/45/problema-testatine-su-classicthesis-style
% ********************************************************************
%\linespread{1.05} % a bit more for Palatino
% ****************************************************************************************************

%%% SYNTAX %%%
\newcommand{\bnf}{::=}
\newcommand{\midd}{\; \; \mbox{\Large{$\mid$}}\;\;}

%%% TERMS %%%
\newcommand{\termone}{M}
\newcommand{\termtwo}{N}
\newcommand{\termthree}{L}
\newcommand{\termfour}{P}
\newcommand{\termfive}{S}
\newcommand{\termsix}{V}
\newcommand{\termseven}{A}
\newcommand{\termeight}{B}
\newcommand{\varone}{x}
\newcommand{\abstr}[2]{\lambda #1.#2}
\newcommand{\subst}[3]{#1\{#2/#3\}}
\newcommand{\rdxone}{R}
\newcommand{\rdxtwo}{Q}
\newcommand{\rdxs}[1]{\mathcal{R}_{#1}}
\newcommand{\rdxset}{\mathcal{R}}
\newcommand{\crdxs}[1]{\mathcal{N}_{#1}}

%%% CONTEXTS %%%
\newcommand{\contone}{C}
\newcommand{\conttwo}{D}
\newcommand{\hole}{[\cdot]}

%%% REDUCTION RELATIONS %%%
\newcommand{\redbeta}{\longrightarrow_\beta}
\newcommand{\redbetaclo}{\xtwoheadrightarrow{}_\beta}
\newcommand{\red}{\rightarrow}
\newcommand{\redlo}{\longrightarrow_\pslo}
\newcommand{\redri}{\longrightarrow_\psri}
\newcommand{\redbetardx}[1]{\overset{#1}{\longrightarrow_\beta}}
\newcommand{\redbetared}[1]{\overset{#1}{\xtwoheadrightarrow{}_\beta}}
\newcommand{\redlosteps}[1]{\longrightarrow_\pslo^#1}
\newcommand{\redlordx}[1]{\overset{#1}{\longrightarrow}_\pslo}
\newcommand{\redanf}{\underset{\mathsf{ANF}}{\longrightarrow}}
\newcommand{\redbetaanf}{\underset{\beta\mathsf{ANF}}{\longrightarrow}}
\newcommand{\redbetaanfsteps}[1]{\underset{\beta\mathsf{ANF}}{\longrightarrow^{#1}}}
\newcommand{\redbetasteps}[1]{\longrightarrow^{#1}_\beta}
\newcommand{\redbetaanfrdx}[1]{\overset{#1}{\underset{\beta\mathsf{ANF}}{\longrightarrow}}}

%%% STRATEGIES %%%
\newcommand{\psone}{\mathsf{P}}
\newcommand{\mcp}[1]{\mathcal{M}_{#1}}
\newcommand{\density}[1]{{f}_{#1}}
\newcommand{\densityp}[2]{{f}_{#1}^{#2}}
\newcommand{\psuni}{\mathsf{U}}
\newcommand{\pslo}{\mathsf{LO}}
\newcommand{\psri}{\mathsf{RI}}
\newcommand{\psparam}[1]{\mathsf{P}(#1)}
\newcommand{\psparamtwo}[1]{\mathsf{P_2}(#1)}
\newcommand{\psgeo}{\mathsf{G}}
\newcommand{\dist}[1]{\mathsf{Dist}\left(#1\right)}
\newcommand{\supp}[1]{\mathbf{Supp}\left(#1\right)}
\newcommand{\pdist}[1]{\mathsf{PDist}\left(#1\right)}
\newcommand{\derlenght}[1]{\mathsf{dl_\mathcal{#1}}}
\newcommand{\avglenght}[1]{\mathsf{adl_\mathcal{#1}}}

%%% MISC %%%
\newcommand{\depth}[2]{d_{#1}(#2)}
\newcommand{\nsteps}[1]{\mathsf{Steps}_{#1}}
\newcommand{\pseries}[1]{\mathsf{PS}(#1)}
\newcommand{\poly}[1]{\mathsf{Poly}(#1)}

\newenvironment{varitemize}
{
	\begin{list}{\labelitemi}
		{\setlength{\itemsep}{0pt}
			\setlength{\topsep}{0pt}
			\setlength{\parsep}{0pt}
			\setlength{\partopsep}{0pt}
			\setlength{\leftmargin}{15pt}
			\setlength{\rightmargin}{0pt}
			\setlength{\itemindent}{0pt}
			\setlength{\labelsep}{5pt}
			\setlength{\labelwidth}{10pt}
		}}
		{
	\end{list}
}

\usepackage{extpfeil}
\usepackage{bm}
\usepackage{ifthen}
\usepackage{amssymb,amsthm}
\usepackage{microtype}
\usepackage{tikz}
\usetikzlibrary{trees}
\usepackage{rotating}
\usetikzlibrary{automata,positioning}
\renewcommand{\labelitemi}{$\bullet$}
%%%  TREES %%%
\tikzstyle{level 1}=[level distance=2.5cm, sibling distance=4.5cm]
\tikzstyle{level 2}=[level distance=2.3cm, sibling distance=4.5cm]
\tikzstyle{term} = [text centered]
\newboolean{lv}
\setboolean{lv}{true}
\newboolean{verbose}
\setboolean{verbose}{true}


%%% MATH %%%
\theoremstyle{plain}
\newtheorem{theorem}{\protect\theoremname}
\theoremstyle{definition}
\newtheorem{definition}[theorem]{\protect\definitionname}
\theoremstyle{definition}
%\newtheorem{example}[theorem]{\protect\examplename}
\theoremstyle{definition}
\newtheorem{lemma}[theorem]{\protect\lemmaname}
\theoremstyle{definition}
\newtheorem{corollary}[theorem]{\protect\corollaryname}
\theoremstyle{definition}
\newtheorem{proposition}[theorem]{\protect\propositionname}
\theoremstyle{definition}
%\newtheorem*{example*}{\protect\examplename}
\newtheorem*{proposition*}{\protect\propositionname}
\newtheorem*{corollary*}{\protect\corollaryname}
\newtheorem*{lemma*}{\protect\lemmaname}
\newtheorem*{theorem*}{\protect\theoremname}
\newtheorem*{remark*}{\protect\remarkname}
\newtheorem{strategy}[theorem]{\protect\strategyname}
%\providecommand{\examplename}{Example}
\providecommand{\strategyname}{Strategy}
\providecommand{\theoremname}{Theorem}
\providecommand{\propositionname}{Proposition}
\providecommand{\corollaryname}{Corollary}
\providecommand{\lemmaname}{Lemma}
\providecommand{\theoremname}{Theorem}
\providecommand{\definitionname}{Definition}
\providecommand{\remarkname}{Remark}
\newtheorem{examplex}[theorem]{Example}
\newenvironment{example}
{\pushQED{\qed}\renewcommand{\qedsymbol}{$\triangle$}\examplex}
{\popQED\endexamplex}
% ****************************************************************************************************
% If you like the classicthesis, then I would appreciate a postcard.
% My address can be found in the file ClassicThesis.pdf. A collection
% of the postcards I received so far is available online at
% http://postcards.miede.de
% ****************************************************************************************************


% ****************************************************************************************************
% 0. Set the encoding of your files. UTF-8 is the only sensible encoding nowadays. If you can't read
% äöüßáéçèê∂åëæƒÏ€ then change the encoding setting in your editor, not the line below. If your editor
% does not support utf8 use another editor!
% ****************************************************************************************************
\PassOptionsToPackage{utf8}{inputenc}
  \usepackage{inputenc}

% ****************************************************************************************************
% 1. Configure classicthesis for your needs here, e.g., remove "drafting" below
% in order to deactivate the time-stamp on the pages
% (see ClassicThesis.pdf for more information):
% ****************************************************************************************************
\PassOptionsToPackage{
  drafting=false,    % print version information on the bottom of the pages
  tocaligned=false, % the left column of the toc will be aligned (no indentation)
  dottedtoc=false,  % page numbers in ToC flushed right
  parts=false,       % use part division
  eulerchapternumbers=true, % use AMS Euler for chapter font (otherwise Palatino)
  linedheaders=true,       % chaper headers will have line above and beneath
  floatperchapter=true,     % numbering per chapter for all floats (i.e., Figure 1.1)
  listings=true,    % load listings package and setup LoL
  subfig=true,      % setup for preloaded subfig package
  eulermath=true,  % use awesome Euler fonts for mathematical formulae (only with pdfLaTeX)
  beramono=true,    % toggle a nice monospaced font (w/ bold)
  minionpro=false   % setup for minion pro font; use minion pro small caps as well (only with pdfLaTeX)
}{classicthesis}


% ****************************************************************************************************
% 2. Personal data and user ad-hoc commands
% ****************************************************************************************************
\newcommand{\myTitle}{On Randomised Strategies in the $\lambda$-calculus\xspace}
\newcommand{\mySubtitle}{}
\newcommand{\myDegree}{Dottor Ingegnere (Dott. Ing.)\xspace}
\newcommand{\myName}{Gabriele Vanoni\xspace}
\newcommand{\myProf}{Matteo Pradella\xspace}
\newcommand{\myOtherProf}{Ugo Dal Lago\xspace}
\newcommand{\mySupervisor}{Put name here\xspace}
\newcommand{\myFaculty}{Scuola di Ingegneria Industriale e dell'Informazione\xspace}
\newcommand{\myDepartment}{Corso di Laurea Magistrale in Ingegneria Informatica\xspace}
\newcommand{\myUni}{Politecnico di Milano\xspace}
\newcommand{\myLocation}{Milano\xspace}
\newcommand{\myTime}{July 2018\xspace}
\newcommand{\myVersion}{version 4.4}

% ********************************************************************
% Setup, finetuning, and useful commands
% ********************************************************************
\newcounter{dummy} % necessary for correct hyperlinks (to index, bib, etc.)
\newlength{\abcd} % for ab..z string length calculation
\providecommand{\mLyX}{L\kern-.1667em\lower.25em\hbox{Y}\kern-.125emX\@}
\newcommand{\ie}{i.\,e.}
\newcommand{\Ie}{I.\,e.}
\newcommand{\eg}{e.\,g.}
\newcommand{\Eg}{E.\,g.}
% ****************************************************************************************************


% ****************************************************************************************************
% 3. Loading some handy packages
% ****************************************************************************************************
% ********************************************************************
% Packages with options that might require adjustments
% ********************************************************************
%\PassOptionsToPackage{ngerman,american}{babel}   % change this to your language(s), main language last
% Spanish languages need extra options in order to work with this template
%\PassOptionsToPackage{spanish,es-lcroman}{babel}
    \usepackage{babel}
\usepackage{csquotes}
\PassOptionsToPackage{%
  %backend=biber,bibencoding=utf8, %instead of bibtex
  backend=bibtex8,bibencoding=ascii,%
  language=auto,%
  style=numeric-comp,%
  %style=authoryear-comp, % Author 1999, 2010
  %citestyle=authoryear,
  %bibstyle=authoryear,dashed=false, % dashed: substitute rep. author with ---
  sorting=nyt, % name, year, title
  maxbibnames=10, % default: 3, et al.
  %backref=true,%
  natbib=true % natbib compatibility mode (\citep and \citet still work)
}{biblatex}
    \usepackage{biblatex}

\PassOptionsToPackage{fleqn}{amsmath}       % math environments and more by the AMS
  \usepackage{amsmath}

% ********************************************************************
% General useful packages
% ********************************************************************
\PassOptionsToPackage{T1}{fontenc} % T2A for cyrillics
  \usepackage{fontenc}
\usepackage{textcomp} % fix warning with missing font shapes
\usepackage{scrhack} % fix warnings when using KOMA with listings package
\usepackage{xspace} % to get the spacing after macros right
\usepackage{mparhack} % get marginpar right
%\usepackage{fixltx2e} % fixes some LaTeX stuff --> since 2015 in the LaTeX kernel (see below)
% \usepackage[latest]{latexrelease} % emulate newer kernel version if older is detected
\PassOptionsToPackage{printonlyused,smaller}{acronym}
  \usepackage{acronym} % nice macros for handling all acronyms in the thesis
  %\renewcommand{\bflabel}[1]{{#1}\hfill} % fix the list of acronyms --> no longer working
  %\renewcommand*{\acsfont}[1]{\textsc{#1}}
  %\renewcommand*{\aclabelfont}[1]{\acsfont{#1}}
  %\def\bflabel#1{{#1\hfill}}
  \def\bflabel#1{{\acsfont{#1}\hfill}}
  \def\aclabelfont#1{\acsfont{#1}}
% ****************************************************************************************************
%\usepackage{pgfplots} % External TikZ/PGF support (thanks to Andreas Nautsch)
%\usetikzlibrary{external}
%\tikzexternalize[mode=list and make, prefix=ext-tikz/]
% ****************************************************************************************************


% ****************************************************************************************************
% 4. Setup floats: tables, (sub)figures, and captions
% ****************************************************************************************************
\usepackage{tabularx} % better tables
  \setlength{\extrarowheight}{3pt} % increase table row height
\newcommand{\tableheadline}[1]{\multicolumn{1}{c}{\spacedlowsmallcaps{#1}}}
\newcommand{\myfloatalign}{\centering} % to be used with each float for alignment
\usepackage{caption}
% Thanks to cgnieder and Claus Lahiri
% http://tex.stackexchange.com/questions/69349/spacedlowsmallcaps-in-caption-label
% [REMOVED DUE TO OTHER PROBLEMS, SEE ISSUE #82]
%\DeclareCaptionLabelFormat{smallcaps}{\bothIfFirst{#1}{~}\MakeTextLowercase{\textsc{#2}}}
%\captionsetup{font=small,labelformat=smallcaps} % format=hang,
\captionsetup{font=small} % format=hang,
\usepackage{subfig}
% ****************************************************************************************************


% ****************************************************************************************************
% 5. Setup code listings
% ****************************************************************************************************
\usepackage{listings}
%\lstset{emph={trueIndex,root},emphstyle=\color{BlueViolet}}%\underbar} % for special keywords
\lstset{language=[LaTeX]Tex,%C++,
  morekeywords={PassOptionsToPackage,selectlanguage},
  keywordstyle=\color{RoyalBlue},%\bfseries,
  basicstyle=\small\ttfamily,
  %identifierstyle=\color{NavyBlue},
  commentstyle=\color{Green}\ttfamily,
  stringstyle=\rmfamily,
  numbers=none,%left,%
  numberstyle=\scriptsize,%\tiny
  stepnumber=5,
  numbersep=8pt,
  showstringspaces=false,
  breaklines=true,
  %frameround=ftff,
  %frame=single,
  belowcaptionskip=.75\baselineskip
  %frame=L
}
% ****************************************************************************************************


% ****************************************************************************************************
% 6. PDFLaTeX, hyperreferences, and citation backreferences
% ****************************************************************************************************
% ********************************************************************
% Using PDFLaTeX
% ********************************************************************
\PassOptionsToPackage{hyperfootnotes=false,pdfpagelabels}{hyperref}
  \usepackage{hyperref}  % backref linktocpage pagebackref
%\ifpdf
%\pdfcompresslevel=9
%\pdfadjustspacing=1
%\fi
%\PassOptionsToPackage{pdftex}{graphicx} %%%IVO: driver will be chosen automatically
  \usepackage{graphicx}


% ********************************************************************
% Hyperreferences
% ********************************************************************
\hypersetup{%
  %draft, % hyperref's draft mode, for printing see below
  colorlinks=true, linktocpage=true, pdfstartpage=3, pdfstartview=FitV,%
  % uncomment the following line if you want to have black links (e.g., for printing)
  %colorlinks=false, linktocpage=false, pdfstartpage=3, pdfstartview=FitV, pdfborder={0 0 0},%
  breaklinks=true, pdfpagemode=UseNone, pageanchor=true, pdfpagemode=UseOutlines,%
  plainpages=false, bookmarksnumbered, bookmarksopen=true, bookmarksopenlevel=1,%
  hypertexnames=true, pdfhighlight=/O,%nesting=true,%frenchlinks,%
  urlcolor=webbrown, linkcolor=RoyalBlue, citecolor=webgreen, %pagecolor=RoyalBlue,%
  %urlcolor=Black, linkcolor=Black, citecolor=Black, %pagecolor=Black,%
  pdftitle={\myTitle},%
  pdfauthor={\textcopyright\ \myName, \myUni, \myFaculty},%
  pdfsubject={},%
  pdfkeywords={},%
  pdfcreator={pdfLaTeX},%
  pdfproducer={LaTeX with hyperref and classicthesis}%
}

% ********************************************************************
% Setup autoreferences
% ********************************************************************
% There are some issues regarding autorefnames
% http://www.ureader.de/msg/136221647.aspx
% http://www.tex.ac.uk/cgi-bin/texfaq2html?label=latexwords
% you have to redefine the makros for the
% language you use, e.g., american, ngerman
% (as chosen when loading babel/AtBeginDocument)
% ********************************************************************
\makeatletter
\@ifpackageloaded{babel}%
  {%
    \addto\extrasamerican{%
      \renewcommand*{\figureautorefname}{Figure}%
      \renewcommand*{\tableautorefname}{Table}%
      \renewcommand*{\partautorefname}{Part}%
      \renewcommand*{\chapterautorefname}{Chapter}%
      \renewcommand*{\sectionautorefname}{Section}%
      \renewcommand*{\subsectionautorefname}{Section}%
      \renewcommand*{\subsubsectionautorefname}{Section}%
    }%
    \addto\extrasngerman{%
      \renewcommand*{\paragraphautorefname}{Absatz}%
      \renewcommand*{\subparagraphautorefname}{Unterabsatz}%
      \renewcommand*{\footnoteautorefname}{Fu\"snote}%
      \renewcommand*{\FancyVerbLineautorefname}{Zeile}%
      \renewcommand*{\theoremautorefname}{Theorem}%
      \renewcommand*{\appendixautorefname}{Anhang}%
      \renewcommand*{\equationautorefname}{Gleichung}%
      \renewcommand*{\itemautorefname}{Punkt}%
    }%
      % Fix to getting autorefs for subfigures right (thanks to Belinda Vogt for changing the definition)
      \providecommand{\subfigureautorefname}{\figureautorefname}%
    }{\relax}
\makeatother


% ****************************************************************************************************
% 7. Last calls before the bar closes
% ****************************************************************************************************
% ********************************************************************
% Development Stuff
% ********************************************************************
\listfiles
%\PassOptionsToPackage{l2tabu,orthodox,abort}{nag}
%  \usepackage{nag}
%\PassOptionsToPackage{warning, all}{onlyamsmath}
%  \usepackage{onlyamsmath}

% ********************************************************************
% Last, but not least...
% ********************************************************************
\usepackage{classicthesis}
% ****************************************************************************************************


% ****************************************************************************************************
% 8. Further adjustments (experimental)
% ****************************************************************************************************
% ********************************************************************
% Changing the text area
% ********************************************************************
%\areaset[current]{312pt}{761pt} % 686 (factor 2.2) + 33 head + 42 head \the\footskip
%\setlength{\marginparwidth}{7em}%
%\setlength{\marginparsep}{2em}%

% ********************************************************************
% Using different fonts
% ********************************************************************
%\usepackage[oldstylenums]{kpfonts} % oldstyle notextcomp
%\usepackage[osf]{libertine}
%\usepackage[light,condensed,math]{iwona}
%\renewcommand{\sfdefault}{iwona}
%\usepackage{lmodern} % <-- no osf support :-(
%\usepackage{cfr-lm} %
%\usepackage[urw-garamond]{mathdesign} <-- no osf support :-(
%\usepackage[default,osfigures]{opensans} % scale=0.95
%\usepackage[sfdefault]{FiraSans}
% ********************************************************************
% \usepackage[largesc,osf]{newpxtext}
% Used to fix these:
% https://bitbucket.org/amiede/classicthesis/issues/139/italics-in-pallatino-capitals-chapter
% https://bitbucket.org/amiede/classicthesis/issues/45/problema-testatine-su-classicthesis-style
% ********************************************************************
%\linespread{1.05} % a bit more for Palatino
% ****************************************************************************************************

%%% SYNTAX %%%
\newcommand{\bnf}{::=}
\newcommand{\midd}{\; \; \mbox{\Large{$\mid$}}\;\;}

%%% TERMS %%%
\newcommand{\termone}{M}
\newcommand{\termtwo}{N}
\newcommand{\termthree}{L}
\newcommand{\termfour}{P}
\newcommand{\termfive}{S}
\newcommand{\termsix}{V}
\newcommand{\termseven}{A}
\newcommand{\termeight}{B}
\newcommand{\varone}{x}
\newcommand{\abstr}[2]{\lambda #1.#2}
\newcommand{\subst}[3]{#1\{#2/#3\}}
\newcommand{\rdxone}{R}
\newcommand{\rdxtwo}{Q}
\newcommand{\rdxs}[1]{\mathcal{R}_{#1}}
\newcommand{\rdxset}{\mathcal{R}}
\newcommand{\crdxs}[1]{\mathcal{N}_{#1}}

%%% CONTEXTS %%%
\newcommand{\contone}{C}
\newcommand{\conttwo}{D}
\newcommand{\hole}{[\cdot]}

%%% REDUCTION RELATIONS %%%
\newcommand{\redbeta}{\longrightarrow_\beta}
\newcommand{\redbetaclo}{\xtwoheadrightarrow{}_\beta}
\newcommand{\red}{\rightarrow}
\newcommand{\redlo}{\longrightarrow_\pslo}
\newcommand{\redri}{\longrightarrow_\psri}
\newcommand{\redbetardx}[1]{\overset{#1}{\longrightarrow_\beta}}
\newcommand{\redbetared}[1]{\overset{#1}{\xtwoheadrightarrow{}_\beta}}
\newcommand{\redlosteps}[1]{\longrightarrow_\pslo^#1}
\newcommand{\redlordx}[1]{\overset{#1}{\longrightarrow}_\pslo}
\newcommand{\redanf}{\underset{\mathsf{ANF}}{\longrightarrow}}
\newcommand{\redbetaanf}{\underset{\beta\mathsf{ANF}}{\longrightarrow}}
\newcommand{\redbetaanfsteps}[1]{\underset{\beta\mathsf{ANF}}{\longrightarrow^{#1}}}
\newcommand{\redbetasteps}[1]{\longrightarrow^{#1}_\beta}
\newcommand{\redbetaanfrdx}[1]{\overset{#1}{\underset{\beta\mathsf{ANF}}{\longrightarrow}}}

%%% STRATEGIES %%%
\newcommand{\psone}{\mathsf{P}}
\newcommand{\mcp}[1]{\mathcal{M}_{#1}}
\newcommand{\density}[1]{{f}_{#1}}
\newcommand{\densityp}[2]{{f}_{#1}^{#2}}
\newcommand{\psuni}{\mathsf{U}}
\newcommand{\pslo}{\mathsf{LO}}
\newcommand{\psri}{\mathsf{RI}}
\newcommand{\psparam}[1]{\mathsf{P}(#1)}
\newcommand{\psparamtwo}[1]{\mathsf{P_2}(#1)}
\newcommand{\psgeo}{\mathsf{G}}
\newcommand{\dist}[1]{\mathsf{Dist}\left(#1\right)}
\newcommand{\supp}[1]{\mathbf{Supp}\left(#1\right)}
\newcommand{\pdist}[1]{\mathsf{PDist}\left(#1\right)}
\newcommand{\derlenght}[1]{\mathsf{dl_\mathcal{#1}}}
\newcommand{\avglenght}[1]{\mathsf{adl_\mathcal{#1}}}

%%% MISC %%%
\newcommand{\depth}[2]{d_{#1}(#2)}
\newcommand{\nsteps}[1]{\mathsf{Steps}_{#1}}
\newcommand{\pseries}[1]{\mathsf{PS}(#1)}
\newcommand{\poly}[1]{\mathsf{Poly}(#1)}

\newenvironment{varitemize}
{
	\begin{list}{\labelitemi}
		{\setlength{\itemsep}{0pt}
			\setlength{\topsep}{0pt}
			\setlength{\parsep}{0pt}
			\setlength{\partopsep}{0pt}
			\setlength{\leftmargin}{15pt}
			\setlength{\rightmargin}{0pt}
			\setlength{\itemindent}{0pt}
			\setlength{\labelsep}{5pt}
			\setlength{\labelwidth}{10pt}
		}}
		{
	\end{list}
}

\usepackage{extpfeil}
\usepackage{bm}
\usepackage{ifthen}
\usepackage{amssymb,amsthm}
\usepackage{microtype}
\usepackage{tikz}
\usetikzlibrary{trees}
\usepackage{rotating}
\usetikzlibrary{automata,positioning}
\renewcommand{\labelitemi}{$\bullet$}
%%%  TREES %%%
\tikzstyle{level 1}=[level distance=2.5cm, sibling distance=4.5cm]
\tikzstyle{level 2}=[level distance=2.3cm, sibling distance=4.5cm]
\tikzstyle{term} = [text centered]
\newboolean{lv}
\setboolean{lv}{true}
\newboolean{verbose}
\setboolean{verbose}{true}


%%% MATH %%%
\theoremstyle{plain}
\newtheorem{theorem}{\protect\theoremname}
\theoremstyle{definition}
\newtheorem{definition}[theorem]{\protect\definitionname}
\theoremstyle{definition}
%\newtheorem{example}[theorem]{\protect\examplename}
\theoremstyle{definition}
\newtheorem{lemma}[theorem]{\protect\lemmaname}
\theoremstyle{definition}
\newtheorem{corollary}[theorem]{\protect\corollaryname}
\theoremstyle{definition}
\newtheorem{proposition}[theorem]{\protect\propositionname}
\theoremstyle{definition}
%\newtheorem*{example*}{\protect\examplename}
\newtheorem*{proposition*}{\protect\propositionname}
\newtheorem*{corollary*}{\protect\corollaryname}
\newtheorem*{lemma*}{\protect\lemmaname}
\newtheorem*{theorem*}{\protect\theoremname}
\newtheorem*{remark*}{\protect\remarkname}
\newtheorem{strategy}[theorem]{\protect\strategyname}
%\providecommand{\examplename}{Example}
\providecommand{\strategyname}{Strategy}
\providecommand{\theoremname}{Theorem}
\providecommand{\propositionname}{Proposition}
\providecommand{\corollaryname}{Corollary}
\providecommand{\lemmaname}{Lemma}
\providecommand{\theoremname}{Theorem}
\providecommand{\definitionname}{Definition}
\providecommand{\remarkname}{Remark}
\newtheorem{examplex}[theorem]{Example}
\newenvironment{example}
{\pushQED{\qed}\renewcommand{\qedsymbol}{$\triangle$}\examplex}
{\popQED\endexamplex}
% ****************************************************************************************************
% If you like the classicthesis, then I would appreciate a postcard.
% My address can be found in the file ClassicThesis.pdf. A collection
% of the postcards I received so far is available online at
% http://postcards.miede.de
% ****************************************************************************************************


% ****************************************************************************************************
% 0. Set the encoding of your files. UTF-8 is the only sensible encoding nowadays. If you can't read
% äöüßáéçèê∂åëæƒÏ€ then change the encoding setting in your editor, not the line below. If your editor
% does not support utf8 use another editor!
% ****************************************************************************************************
\PassOptionsToPackage{utf8}{inputenc}
  \usepackage{inputenc}

% ****************************************************************************************************
% 1. Configure classicthesis for your needs here, e.g., remove "drafting" below
% in order to deactivate the time-stamp on the pages
% (see ClassicThesis.pdf for more information):
% ****************************************************************************************************
\PassOptionsToPackage{
  drafting=false,    % print version information on the bottom of the pages
  tocaligned=false, % the left column of the toc will be aligned (no indentation)
  dottedtoc=false,  % page numbers in ToC flushed right
  parts=false,       % use part division
  eulerchapternumbers=true, % use AMS Euler for chapter font (otherwise Palatino)
  linedheaders=true,       % chaper headers will have line above and beneath
  floatperchapter=true,     % numbering per chapter for all floats (i.e., Figure 1.1)
  listings=true,    % load listings package and setup LoL
  subfig=true,      % setup for preloaded subfig package
  eulermath=true,  % use awesome Euler fonts for mathematical formulae (only with pdfLaTeX)
  beramono=true,    % toggle a nice monospaced font (w/ bold)
  minionpro=false   % setup for minion pro font; use minion pro small caps as well (only with pdfLaTeX)
}{classicthesis}


% ****************************************************************************************************
% 2. Personal data and user ad-hoc commands
% ****************************************************************************************************
\newcommand{\myTitle}{On Randomised Strategies in the $\lambda$-calculus\xspace}
\newcommand{\mySubtitle}{}
\newcommand{\myDegree}{Dottor Ingegnere (Dott. Ing.)\xspace}
\newcommand{\myName}{Gabriele Vanoni\xspace}
\newcommand{\myProf}{Matteo Pradella\xspace}
\newcommand{\myOtherProf}{Ugo Dal Lago\xspace}
\newcommand{\mySupervisor}{Put name here\xspace}
\newcommand{\myFaculty}{Scuola di Ingegneria Industriale e dell'Informazione\xspace}
\newcommand{\myDepartment}{Corso di Laurea Magistrale in Ingegneria Informatica\xspace}
\newcommand{\myUni}{Politecnico di Milano\xspace}
\newcommand{\myLocation}{Milano\xspace}
\newcommand{\myTime}{July 2018\xspace}
\newcommand{\myVersion}{version 4.4}

% ********************************************************************
% Setup, finetuning, and useful commands
% ********************************************************************
\newcounter{dummy} % necessary for correct hyperlinks (to index, bib, etc.)
\newlength{\abcd} % for ab..z string length calculation
\providecommand{\mLyX}{L\kern-.1667em\lower.25em\hbox{Y}\kern-.125emX\@}
\newcommand{\ie}{i.\,e.}
\newcommand{\Ie}{I.\,e.}
\newcommand{\eg}{e.\,g.}
\newcommand{\Eg}{E.\,g.}
% ****************************************************************************************************


% ****************************************************************************************************
% 3. Loading some handy packages
% ****************************************************************************************************
% ********************************************************************
% Packages with options that might require adjustments
% ********************************************************************
%\PassOptionsToPackage{ngerman,american}{babel}   % change this to your language(s), main language last
% Spanish languages need extra options in order to work with this template
%\PassOptionsToPackage{spanish,es-lcroman}{babel}
    \usepackage{babel}
\usepackage{csquotes}
\PassOptionsToPackage{%
  %backend=biber,bibencoding=utf8, %instead of bibtex
  backend=bibtex8,bibencoding=ascii,%
  language=auto,%
  style=numeric-comp,%
  %style=authoryear-comp, % Author 1999, 2010
  %citestyle=authoryear,
  %bibstyle=authoryear,dashed=false, % dashed: substitute rep. author with ---
  sorting=nyt, % name, year, title
  maxbibnames=10, % default: 3, et al.
  %backref=true,%
  natbib=true % natbib compatibility mode (\citep and \citet still work)
}{biblatex}
    \usepackage{biblatex}

\PassOptionsToPackage{fleqn}{amsmath}       % math environments and more by the AMS
  \usepackage{amsmath}

% ********************************************************************
% General useful packages
% ********************************************************************
\PassOptionsToPackage{T1}{fontenc} % T2A for cyrillics
  \usepackage{fontenc}
\usepackage{textcomp} % fix warning with missing font shapes
\usepackage{scrhack} % fix warnings when using KOMA with listings package
\usepackage{xspace} % to get the spacing after macros right
\usepackage{mparhack} % get marginpar right
%\usepackage{fixltx2e} % fixes some LaTeX stuff --> since 2015 in the LaTeX kernel (see below)
% \usepackage[latest]{latexrelease} % emulate newer kernel version if older is detected
\PassOptionsToPackage{printonlyused,smaller}{acronym}
  \usepackage{acronym} % nice macros for handling all acronyms in the thesis
  %\renewcommand{\bflabel}[1]{{#1}\hfill} % fix the list of acronyms --> no longer working
  %\renewcommand*{\acsfont}[1]{\textsc{#1}}
  %\renewcommand*{\aclabelfont}[1]{\acsfont{#1}}
  %\def\bflabel#1{{#1\hfill}}
  \def\bflabel#1{{\acsfont{#1}\hfill}}
  \def\aclabelfont#1{\acsfont{#1}}
% ****************************************************************************************************
%\usepackage{pgfplots} % External TikZ/PGF support (thanks to Andreas Nautsch)
%\usetikzlibrary{external}
%\tikzexternalize[mode=list and make, prefix=ext-tikz/]
% ****************************************************************************************************


% ****************************************************************************************************
% 4. Setup floats: tables, (sub)figures, and captions
% ****************************************************************************************************
\usepackage{tabularx} % better tables
  \setlength{\extrarowheight}{3pt} % increase table row height
\newcommand{\tableheadline}[1]{\multicolumn{1}{c}{\spacedlowsmallcaps{#1}}}
\newcommand{\myfloatalign}{\centering} % to be used with each float for alignment
\usepackage{caption}
% Thanks to cgnieder and Claus Lahiri
% http://tex.stackexchange.com/questions/69349/spacedlowsmallcaps-in-caption-label
% [REMOVED DUE TO OTHER PROBLEMS, SEE ISSUE #82]
%\DeclareCaptionLabelFormat{smallcaps}{\bothIfFirst{#1}{~}\MakeTextLowercase{\textsc{#2}}}
%\captionsetup{font=small,labelformat=smallcaps} % format=hang,
\captionsetup{font=small} % format=hang,
\usepackage{subfig}
% ****************************************************************************************************


% ****************************************************************************************************
% 5. Setup code listings
% ****************************************************************************************************
\usepackage{listings}
%\lstset{emph={trueIndex,root},emphstyle=\color{BlueViolet}}%\underbar} % for special keywords
\lstset{language=[LaTeX]Tex,%C++,
  morekeywords={PassOptionsToPackage,selectlanguage},
  keywordstyle=\color{RoyalBlue},%\bfseries,
  basicstyle=\small\ttfamily,
  %identifierstyle=\color{NavyBlue},
  commentstyle=\color{Green}\ttfamily,
  stringstyle=\rmfamily,
  numbers=none,%left,%
  numberstyle=\scriptsize,%\tiny
  stepnumber=5,
  numbersep=8pt,
  showstringspaces=false,
  breaklines=true,
  %frameround=ftff,
  %frame=single,
  belowcaptionskip=.75\baselineskip
  %frame=L
}
% ****************************************************************************************************


% ****************************************************************************************************
% 6. PDFLaTeX, hyperreferences, and citation backreferences
% ****************************************************************************************************
% ********************************************************************
% Using PDFLaTeX
% ********************************************************************
\PassOptionsToPackage{hyperfootnotes=false,pdfpagelabels}{hyperref}
  \usepackage{hyperref}  % backref linktocpage pagebackref
%\ifpdf
%\pdfcompresslevel=9
%\pdfadjustspacing=1
%\fi
%\PassOptionsToPackage{pdftex}{graphicx} %%%IVO: driver will be chosen automatically
  \usepackage{graphicx}


% ********************************************************************
% Hyperreferences
% ********************************************************************
\hypersetup{%
  %draft, % hyperref's draft mode, for printing see below
  colorlinks=true, linktocpage=true, pdfstartpage=3, pdfstartview=FitV,%
  % uncomment the following line if you want to have black links (e.g., for printing)
  %colorlinks=false, linktocpage=false, pdfstartpage=3, pdfstartview=FitV, pdfborder={0 0 0},%
  breaklinks=true, pdfpagemode=UseNone, pageanchor=true, pdfpagemode=UseOutlines,%
  plainpages=false, bookmarksnumbered, bookmarksopen=true, bookmarksopenlevel=1,%
  hypertexnames=true, pdfhighlight=/O,%nesting=true,%frenchlinks,%
  urlcolor=webbrown, linkcolor=RoyalBlue, citecolor=webgreen, %pagecolor=RoyalBlue,%
  %urlcolor=Black, linkcolor=Black, citecolor=Black, %pagecolor=Black,%
  pdftitle={\myTitle},%
  pdfauthor={\textcopyright\ \myName, \myUni, \myFaculty},%
  pdfsubject={},%
  pdfkeywords={},%
  pdfcreator={pdfLaTeX},%
  pdfproducer={LaTeX with hyperref and classicthesis}%
}

% ********************************************************************
% Setup autoreferences
% ********************************************************************
% There are some issues regarding autorefnames
% http://www.ureader.de/msg/136221647.aspx
% http://www.tex.ac.uk/cgi-bin/texfaq2html?label=latexwords
% you have to redefine the makros for the
% language you use, e.g., american, ngerman
% (as chosen when loading babel/AtBeginDocument)
% ********************************************************************
\makeatletter
\@ifpackageloaded{babel}%
  {%
    \addto\extrasamerican{%
      \renewcommand*{\figureautorefname}{Figure}%
      \renewcommand*{\tableautorefname}{Table}%
      \renewcommand*{\partautorefname}{Part}%
      \renewcommand*{\chapterautorefname}{Chapter}%
      \renewcommand*{\sectionautorefname}{Section}%
      \renewcommand*{\subsectionautorefname}{Section}%
      \renewcommand*{\subsubsectionautorefname}{Section}%
    }%
    \addto\extrasngerman{%
      \renewcommand*{\paragraphautorefname}{Absatz}%
      \renewcommand*{\subparagraphautorefname}{Unterabsatz}%
      \renewcommand*{\footnoteautorefname}{Fu\"snote}%
      \renewcommand*{\FancyVerbLineautorefname}{Zeile}%
      \renewcommand*{\theoremautorefname}{Theorem}%
      \renewcommand*{\appendixautorefname}{Anhang}%
      \renewcommand*{\equationautorefname}{Gleichung}%
      \renewcommand*{\itemautorefname}{Punkt}%
    }%
      % Fix to getting autorefs for subfigures right (thanks to Belinda Vogt for changing the definition)
      \providecommand{\subfigureautorefname}{\figureautorefname}%
    }{\relax}
\makeatother


% ****************************************************************************************************
% 7. Last calls before the bar closes
% ****************************************************************************************************
% ********************************************************************
% Development Stuff
% ********************************************************************
\listfiles
%\PassOptionsToPackage{l2tabu,orthodox,abort}{nag}
%  \usepackage{nag}
%\PassOptionsToPackage{warning, all}{onlyamsmath}
%  \usepackage{onlyamsmath}

% ********************************************************************
% Last, but not least...
% ********************************************************************
\usepackage{classicthesis}
% ****************************************************************************************************


% ****************************************************************************************************
% 8. Further adjustments (experimental)
% ****************************************************************************************************
% ********************************************************************
% Changing the text area
% ********************************************************************
%\areaset[current]{312pt}{761pt} % 686 (factor 2.2) + 33 head + 42 head \the\footskip
%\setlength{\marginparwidth}{7em}%
%\setlength{\marginparsep}{2em}%

% ********************************************************************
% Using different fonts
% ********************************************************************
%\usepackage[oldstylenums]{kpfonts} % oldstyle notextcomp
%\usepackage[osf]{libertine}
%\usepackage[light,condensed,math]{iwona}
%\renewcommand{\sfdefault}{iwona}
%\usepackage{lmodern} % <-- no osf support :-(
%\usepackage{cfr-lm} %
%\usepackage[urw-garamond]{mathdesign} <-- no osf support :-(
%\usepackage[default,osfigures]{opensans} % scale=0.95
%\usepackage[sfdefault]{FiraSans}
% ********************************************************************
% \usepackage[largesc,osf]{newpxtext}
% Used to fix these:
% https://bitbucket.org/amiede/classicthesis/issues/139/italics-in-pallatino-capitals-chapter
% https://bitbucket.org/amiede/classicthesis/issues/45/problema-testatine-su-classicthesis-style
% ********************************************************************
%\linespread{1.05} % a bit more for Palatino
% ****************************************************************************************************

%%% SYNTAX %%%
\newcommand{\bnf}{::=}
\newcommand{\midd}{\; \; \mbox{\Large{$\mid$}}\;\;}

%%% TERMS %%%
\newcommand{\termone}{M}
\newcommand{\termtwo}{N}
\newcommand{\termthree}{L}
\newcommand{\termfour}{P}
\newcommand{\termfive}{S}
\newcommand{\termsix}{V}
\newcommand{\termseven}{A}
\newcommand{\termeight}{B}
\newcommand{\varone}{x}
\newcommand{\abstr}[2]{\lambda #1.#2}
\newcommand{\subst}[3]{#1\{#2/#3\}}
\newcommand{\rdxone}{R}
\newcommand{\rdxtwo}{Q}
\newcommand{\rdxs}[1]{\mathcal{R}_{#1}}
\newcommand{\rdxset}{\mathcal{R}}
\newcommand{\crdxs}[1]{\mathcal{N}_{#1}}

%%% CONTEXTS %%%
\newcommand{\contone}{C}
\newcommand{\conttwo}{D}
\newcommand{\hole}{[\cdot]}

%%% REDUCTION RELATIONS %%%
\newcommand{\redbeta}{\longrightarrow_\beta}
\newcommand{\redbetaclo}{\xtwoheadrightarrow{}_\beta}
\newcommand{\red}{\rightarrow}
\newcommand{\redlo}{\longrightarrow_\pslo}
\newcommand{\redri}{\longrightarrow_\psri}
\newcommand{\redbetardx}[1]{\overset{#1}{\longrightarrow_\beta}}
\newcommand{\redbetared}[1]{\overset{#1}{\xtwoheadrightarrow{}_\beta}}
\newcommand{\redlosteps}[1]{\longrightarrow_\pslo^#1}
\newcommand{\redlordx}[1]{\overset{#1}{\longrightarrow}_\pslo}
\newcommand{\redanf}{\underset{\mathsf{ANF}}{\longrightarrow}}
\newcommand{\redbetaanf}{\underset{\beta\mathsf{ANF}}{\longrightarrow}}
\newcommand{\redbetaanfsteps}[1]{\underset{\beta\mathsf{ANF}}{\longrightarrow^{#1}}}
\newcommand{\redbetasteps}[1]{\longrightarrow^{#1}_\beta}
\newcommand{\redbetaanfrdx}[1]{\overset{#1}{\underset{\beta\mathsf{ANF}}{\longrightarrow}}}

%%% STRATEGIES %%%
\newcommand{\psone}{\mathsf{P}}
\newcommand{\mcp}[1]{\mathcal{M}_{#1}}
\newcommand{\density}[1]{{f}_{#1}}
\newcommand{\densityp}[2]{{f}_{#1}^{#2}}
\newcommand{\psuni}{\mathsf{U}}
\newcommand{\pslo}{\mathsf{LO}}
\newcommand{\psri}{\mathsf{RI}}
\newcommand{\psparam}[1]{\mathsf{P}(#1)}
\newcommand{\psparamtwo}[1]{\mathsf{P_2}(#1)}
\newcommand{\psgeo}{\mathsf{G}}
\newcommand{\dist}[1]{\mathsf{Dist}\left(#1\right)}
\newcommand{\supp}[1]{\mathbf{Supp}\left(#1\right)}
\newcommand{\pdist}[1]{\mathsf{PDist}\left(#1\right)}
\newcommand{\derlenght}[1]{\mathsf{dl_\mathcal{#1}}}
\newcommand{\avglenght}[1]{\mathsf{adl_\mathcal{#1}}}

%%% MISC %%%
\newcommand{\depth}[2]{d_{#1}(#2)}
\newcommand{\nsteps}[1]{\mathsf{Steps}_{#1}}
\newcommand{\pseries}[1]{\mathsf{PS}(#1)}
\newcommand{\poly}[1]{\mathsf{Poly}(#1)}

\newenvironment{varitemize}
{
	\begin{list}{\labelitemi}
		{\setlength{\itemsep}{0pt}
			\setlength{\topsep}{0pt}
			\setlength{\parsep}{0pt}
			\setlength{\partopsep}{0pt}
			\setlength{\leftmargin}{15pt}
			\setlength{\rightmargin}{0pt}
			\setlength{\itemindent}{0pt}
			\setlength{\labelsep}{5pt}
			\setlength{\labelwidth}{10pt}
		}}
		{
	\end{list}
}

\usepackage{extpfeil}
\usepackage{bm}
\usepackage{ifthen}
\usepackage{amssymb,amsthm}
\usepackage{microtype}
\usepackage{tikz}
\usetikzlibrary{trees}
\usepackage{rotating}
\usetikzlibrary{automata,positioning}
\renewcommand{\labelitemi}{$\bullet$}
%%%  TREES %%%
\tikzstyle{level 1}=[level distance=2.5cm, sibling distance=4.5cm]
\tikzstyle{level 2}=[level distance=2.3cm, sibling distance=4.5cm]
\tikzstyle{term} = [text centered]
\newboolean{lv}
\setboolean{lv}{true}
\newboolean{verbose}
\setboolean{verbose}{true}


%%% MATH %%%
\theoremstyle{plain}
\newtheorem{theorem}{\protect\theoremname}
\theoremstyle{definition}
\newtheorem{definition}[theorem]{\protect\definitionname}
\theoremstyle{definition}
%\newtheorem{example}[theorem]{\protect\examplename}
\theoremstyle{definition}
\newtheorem{lemma}[theorem]{\protect\lemmaname}
\theoremstyle{definition}
\newtheorem{corollary}[theorem]{\protect\corollaryname}
\theoremstyle{definition}
\newtheorem{proposition}[theorem]{\protect\propositionname}
\theoremstyle{definition}
%\newtheorem*{example*}{\protect\examplename}
\newtheorem*{proposition*}{\protect\propositionname}
\newtheorem*{corollary*}{\protect\corollaryname}
\newtheorem*{lemma*}{\protect\lemmaname}
\newtheorem*{theorem*}{\protect\theoremname}
\newtheorem*{remark*}{\protect\remarkname}
\newtheorem{strategy}[theorem]{\protect\strategyname}
%\providecommand{\examplename}{Example}
\providecommand{\strategyname}{Strategy}
\providecommand{\theoremname}{Theorem}
\providecommand{\propositionname}{Proposition}
\providecommand{\corollaryname}{Corollary}
\providecommand{\lemmaname}{Lemma}
\providecommand{\theoremname}{Theorem}
\providecommand{\definitionname}{Definition}
\providecommand{\remarkname}{Remark}
\newtheorem{examplex}[theorem]{Example}
\newenvironment{example}
{\pushQED{\qed}\renewcommand{\qedsymbol}{$\triangle$}\examplex}
{\popQED\endexamplex}